\documentclass[titlepage]{article}
\usepackage[ukrainian]{babel}
\usepackage{listings}
\usepackage{amsmath}
\usepackage{fontspec}
\setmainfont{Times New Roman}
\usepackage{setspace}
%\spacing{1.7}
\usepackage{fancyhdr}
\usepackage{titling}
\usepackage[a4paper, top = 20mm, bottom = 20mm, left = 25mm, right = 15mm]{geometry}
\usepackage{graphicx}

\lstset{
	basicstyle=\large,
	breaklines = true,
	language = matlab,
	breakatwhitespace = true,
	numbers = left,
	numberstyle = \tiny,
	columns = flexible,
	frame=tb,
	tabsize=4
}

\makeatletter
\renewcommand\normalsize{%
\@setfontsize\normalsize{14pt}{15pt}}
\makeatother

\preauthor{\begin{flushright}}
\postauthor{\end{flushright}}

\fancypagestyle{empty}{%
	\renewcommand{\headrulewidth}{0pt}
	\cfoot{\normalsize2016}
	\chead{\uppercase{київський національний університет імені тараса шевченка}}
}


\begin{document}

\title{\normalsize Лабораторна робота \textnumero 4}
\author{\vspace{5cm}Виконав: \protect\\ студент 4-го курсу \protect\\ спеціальність Математика \protect\\ Шатохін Михайло}
\date{}
\maketitle
\section{Постановка задачі}
Необхідно обчислити інтеграл $I=\int_{\alpha}^{\beta}{\!\rho(x)U(x)\,dx}$ за допомогою квадратурної формули:
\[I=\int_{a}^{b}{\!\rho(x)U(x)\,dx} \approx \sum_{k=1}^{n}{A_kf(x_k)}\]
В даній задачі:
\begin{description}
\item $\rho(x) =  1 + x$,
\item $[a,b] = [0,1]$,
\item $n = 7$,
\item $[\alpha, \beta] = [7, 21]$,
\item $U(x) = e^\frac{-x}{21+x}$.
\end{description}
\section{Практична реалізація}
У функції main відбувається ініціалізація параметрів задачі, виклик основної частини алгоритму та виведення результату.
\lstinputlisting[title=main.m, language=octave]{./main.m}
У функції GetInterpolationCoefficients відбувається пошук квадратурної формули найвищого степеня точності.
\lstinputlisting[title=GetInterpolationCoefficients.m, language=octave]{./GetInterpolationCoefficients.m}
У функції Interpolate відбувається підрахунок інтеграла за допомогою квадратурної формули.
\lstinputlisting[title=Interpolate.m, language=octave]{./Interpolate.m}
Функція SimpsonsMethod здійснює обрахунок інтеграла методом Сімпсона.
\lstinputlisting[title=SimpsonsMethod.m, language=octave]{./SimpsonsMethod.m}
Файл params.mat містить параметри задачі. Вони наведені у зрозумілому людині вигляді.
\lstinputlisting[title=params.mat, language=octave]{./params.mat}
В результаті роботи програми отримуємо такий результат:
\lstinputlisting[title=output.txt, language=octave]{./output.txt}
\end{document}